\usepackage[table]{xcolor}
\definecolor{mygreen}{RGB}{0, 90, 0}
\newcommand\original[1]{\textcolor{red}{#1}}
\newcommand\TODO[1]{\textcolor{red}{[TODO: #1]}}

%%%%%% EDITING MACROS %%%%%%%%%

\newcommand\new[1]{\textcolor{blue}{#1}}
\newcommand\old[1]{\sout{#1}}
% orange for EI
\definecolor{orange}{rgb}{1,0.5,0}
\newcommand\ei[2]{\sout{#1} \textcolor{orange}{#2}}
\newcommand\eic[1]{\textcolor{orange}{[#1]}}
% green for JW
% \definecolor{green}{rgb}{0,0.5,0}
\newcommand\jw[2]{\sout{#1} \textcolor{purple}{#2}}
\newcommand\jwc[1]{\textcolor{purple}{[#1]}}
% purple for JJ
\definecolor{purple}{rgb}{0.5,0,1}
\newcommand\jj[2]{\sout{#1} \textcolor{purple}{#2}}
\newcommand\jjc[1]{\textcolor{purple}{[Josh: #1]}}
% cyan for AR
\definecolor{cyan}{rgb}{0,.5,.5}
\newcommand\ar[2]{\sout{#1} \textcolor{cyan}{#2}}
\newcommand\arc[1]{\textcolor{cyan}{#1}}
% light brown for KT
\definecolor{lightbrown}{rgb}{0.5,0.5,0}
\newcommand\kt[2]{\sout{#1} \textcolor{lightbrown}{#2}}
\newcommand\ktc[1]{\textcolor{lightbrown}{#1}}

%They recommend five guiding principles: stakeholder engagement, complete model documentation, complete description of data used, communicating uncerainty, and testable model outcomes. 
\newcommand\editModelingGuidanceFirst{There are many guidelines available regarding the use of mechanistic models for policy influence. For instance, Behrend et al.~\cite{behrend20} conducted an extensive literature review on modeling principles and standards. They used this review to develop general recommendations for policy-guiding modeling. This review is particularly relevant given that their recommendations are specifically intended for the modeling of neglected tropical diseases, which includes cholera. Their resulting recommendations for communicating model results align with other manifestos on the proper use of models intended to inform policy \cite{saltelli20,donnelly18}, each emphasizing the importance of involving stakeholders in the modeling process, transparency, and communicating uncertainty. Other recommendations that focus on issues of model calibration and assessment also exist \cite{dahabreh17,egger17,penaloza15}, demonstrating the considerable amount of published guidance on effectively modeling dynamic systems.}

\newcommand\editModelingGuidanceSecond{Based on our assessment and literature review, \cite{lee20} was among the studies that best followed these general recommendations for policy-driven modeling. Concretly, \cite{lee20} follows at least four of the five principles outlined in \cite{behrend20}: \emph{complete model documentation}, \emph{complete description of data used}, \emph{communicating uncertainty}, and \emph{testable model outcomes}. Determining the level of adherence to the first principle, \emph{stakeholder engagement}, is difficult based solely on the article. Despite this, the inconsistency between their forecasts and the cholera incidence from 2019 to 2022 suggests that existing recommendations and standards related to policy-driven modeling may be insufficient. \cite{saltelli19} suggests that improvements in model model-based outputs may be obtained by developing structures and standards based on statistical principles. Our general recommendations therefore complement and extend existing guidelines by focusing on the methodological tasks of calibrating and evaluating dynamic models in a rigorous statistical framework. 
}

\newcommand\editMechModels{Mechanistic models representing biological phenomena are valuable for epidemiology and consequently for public health policy \cite{lofgren14,mccabe21}. More broadly, they have useful roles throughout biology, expecially when combined with statistical methods that properly account for stochasticity and nonlinearity \cite{may04}. In some situations, modern machine learning methods can outperform mechanistic models on epidemiological forecasting tasks \cite{lau22}. The predictive skill of non-mechansitic models can reveal limitations in mechanistic models, but cannot readily replace the scientific understanding obtained by describing the biological dynamics of the system in a mathematical model.}

