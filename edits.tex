\usepackage[table]{xcolor}
\definecolor{mygreen}{RGB}{0, 90, 0}
\newcommand\original[1]{\textcolor{red}{#1}}
\newcommand\TODO[1]{\textcolor{red}{[TODO: #1]}}

%%%%%% EDITING MACROS %%%%%%%%%

\newcommand\new[1]{\textcolor{blue}{#1}}
\newcommand\old[1]{\sout{#1}}
% orange for EI
\definecolor{orange}{rgb}{1,0.5,0}
\newcommand\ei[2]{\sout{#1} \textcolor{orange}{#2}}
\newcommand\eic[1]{\textcolor{orange}{[#1]}}
% green for JW
\definecolor{green}{rgb}{0,0.5,0}
\newcommand\jw[2]{\sout{#1} \textcolor{green}{#2}}
\newcommand\jwc[1]{\textcolor{green}{[#1]}}
% purple for JJ
\definecolor{purple}{rgb}{0.5,0,1}
\newcommand\jj[2]{\sout{#1} \textcolor{purple}{#2}}
\newcommand\jjc[1]{\textcolor{purple}{[#1]}}
% cyan for AR
\definecolor{cyan}{rgb}{0,.5,.5}
\newcommand\ar[2]{\sout{#1} \textcolor{cyan}{#2}}
\newcommand\arc[1]{\textcolor{cyan}{#1}}
% light brown for KT
\definecolor{lightbrown}{rgb}{0.5,0.5,0}
\newcommand\kt[2]{\sout{#1} \textcolor{lightbrown}{#2}}
\newcommand\ktc[1]{\textcolor{lightbrown}{#1}}


\newcommand\editMechModels{Mathematical models representing biological phenomena are valuable for epidemiology and consequently for public health policy \cite{lofgren14,mccabe21}. More broadly, they have useful roles throughout biology, expecially when combined with statistical methods that properly account for stochasticity and nonlinearity \cite{may04}. In some situations, modern machine learning methods can outperform mechanistic models on epidemiological forecasting tasks \cite{lau22}. The predictive skill of non-mechansitic models can reveal limitations in mechanistic models, but cannot readily replace the scientific understanding obtained by describing the biological dynamics of the system in a mathematical model.}

